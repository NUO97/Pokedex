\documentclass[programming]{../../../../mfcs}

% Patch listings package to output proper ascii hyphens in code listings
% https://tex.stackexchange.com/questions/33185/listings-package-changes-hyphens-to-minus-signs
\makeatletter
\lst@CCPutMacro\lst@ProcessOther {"2D}{\lst@ttfamily{-{}}{-{}}}
\@empty\z@\@empty
\makeatother
%

\newcommand{\quarter}{Spring 2017}
\usepackage{color}
\usepackage{multicol}

\course{CSE 154}{Web Programming}{\quarter}
\NoDate
\topic{Homework 5 API Documentation}
\usepackage{csquotes}
\usepackage{tabularx}

\begin{document}
\vspace{-3.8em}

\hfill\begin{varwidth}{0.5\textwidth}
\end{varwidth}
\vspace{1em}

\begin{question}{Overview}
We have provided two web services for you two use on Homework 5: a Pokedex API and a Game Management
API. The Pokedex API provides data about each of the 151 Pokemon, including moves, type, and
weakness. Each type of query produces output in plain text or JSON format (\emph{You can test queries by typing in their URL in your
browser's address bar and seeing the result}). If you submit an invalid query, such as one missing a
necessary parameter, the request will return an HTTP error code of 400 (Invalid request) rather than
the default 200. 
\newline

The rest of this document provides the necessary information about the endpoints and query types for
the requests you will make for HW 5.
\end{question}

\vspace{2em}
\begin{question}{Pokedex API}
\textbf{Endpoint:} \texttt{\textbf{\color{colour}{https://webster.cs.washington.edu/pokedex/pokedex.php}}}
\newline

The first web service, \texttt{pokedex.php}, provides data about each of the 151 Pokemon and accepts two different types of ``GET'' queries, specified using a query string with a parameter.
\newline

\subquestion*{Query 1: Get Pokemon Names}
  \textbf{Request Format:} \url{pokedex.php?pokedex=all}
  \newline
  \textbf{Request Type:} \texttt{GET}
  \newline
  \textbf{Returned Data Format:} \texttt{plain text}
  \newline
  \textbf{Description:} This first request takes the parameter \texttt{all} and returns a plain text
  response with all 151 Pokemon names and sprite image names, each on its own line. The Pokemon name is
  followed by its sprite image name separated by a single ``:''. These sprite image names correspond
  to the Pokemon's sprite image in the \texttt{https://webster.cs.washington.edu/pokedex/sprites/}
  folder.
  \newline
  \textbf{Request:}
  \url{https://webster.cs.washington.edu/pokedex/pokedex.php?pokedex=all}
  \newline
  \textbf{Output:} (abbreviated)
\newline
\hrule
\begin{lstlisting}
Abra:abra.png
Aerodactyl:aerodactyl.png
Alakazam:alakazam.png
...
Zubat:zubat.png
\end{lstlisting}
\hrule
\newpage
\subquestion*{Query 2: Get Pokemon Data}
  \textbf{Request Format:} \url{pokedex.php?pokemon={name}}
  \newline
  \textbf{Request Type:} \texttt{GET}
  \newline
  \textbf{Returned Data Format:} \texttt{JSON}
  \newline
  \textbf{Description:} The second request type takes as a parameter any Pokemon name and returns
  a detailed JSON object containing data about this Pokemon. The returned data will be used to
  populate a card for that Pokemon. 
  \newline
  \textbf{Example Request:} \url{https://webster.cs.washington.edu/pokedex/pokedex.php?pokemon=pikachu}
  \newline
  \textbf{Example Output:}
\vspace{1em}

\hrule
    \begin{lstlisting}
{
  "name": "Pikachu",
  "hp": 160,
  "info": {
    "id": "25",
    "type": "electric",
    "weakness": "ground",
    "description": "Melissa's favorite Pokemon! When several Pikachu gather, their electricity could build and cause lightning storms."
  },
  "images": {
    "photo": "images/pikachu.jpg",
    "typeIcon": "icons/electric.jpg",
    "weaknessIcon": "icons/ground.jpg"
  },
  "moves": [
    {
      "name": "Growl",
      "type": "normal"
    },
    {
      "name": "Quick Attack",
      "dp": 40,
      "type": "normal"
    },
    {
      "name": "Thunderbolt",
      "dp": 90,
      "type": "electric"
    }
  ]
}
\end{lstlisting}
\hrule
\vspace{1em}
The values of the returned JSON object include the name of the Pokemon (e.g., Pikachu), the type of
the Pokmeon (e.g., ``electric''), its weakness type (e.g., ``ground''), the health points, or hp
(e.g., 80), the set of images (photo for the main Pokemon image and \texttt{typeIcon} and
\texttt{weaknessIcon} for the type and weakness icon image paths, respectively), and the set of moves
(each Pokemon has between 1 and 4 moves; Pikachu has 4). Each move has a type, and moves that do
damage have a ``dp'', or damage point attribute. Moves that do not have a ``dp'' attribute (e.g.,
Growl) affect stats of the player or opponent's Pokemon, which is handled elsewhere in the program
(during the game mode).
\end{question}
\newpage
\begin{question}{Game Management API}
\textbf{Endpoint:} \texttt{\textbf{\color{colour}{https://webster.cs.washington.edu/pokedex/game.php}}}
\newline
  The second web service, \texttt{game.php}, accepts two \texttt{POST} query types to initiate and
  update the state of a card game. 
\newline

\subquestion*{Query 3: Start Game}
  \textbf{Request Format:} \url{game.php} endpoint with \texttt{POST} parameters of
  \texttt{startgame} (set to true) and \texttt{mypokemon}
  \newline
  \textbf{Request Type:} \texttt{POST}
  \newline
  \textbf{Returned Data Format:} \texttt{JSON}
  \newline
  \textbf{Description:} The third request you will use initiates a game and passes two parameters,
  \texttt{startgame} and \texttt{mypokemon} to \texttt{game.php}. In contrast to the
  first two ``GET'' requests, this request is a ``\texttt{POST}'' request. Upon success, the request returns
  a JSON response of the initial game state (with information for both players' Pokemon) and unique game id (guid) and player id (pid) for the player to use to access and update the current game
state. 
\newline
\textbf{Example Request:}
\newline
\texttt{POST} parameters of \texttt{startgame=true} and \texttt{mypokemon=pikachu}
\newline
\textbf{Example Output:}
\newline

\hrule
\begin{lstlisting}
{
  "guid" : "game_12345abc",
  "pid" : "player_cfe67890",
  "p1" : {
    "name" : "Pikachu",
    "hp" : 160,
    "current-hp" : 160,
    "images" : {
      "photo" : "images/pikachu.jpg",
      "typeIcon" : "icons/electric.jpg",
      "weaknessIcon" : "icons/ground.jpg"
    },
    "info": {
      "id": "25",
      "type": "electric",
      "weakness": "ground",
      "description": "Melissa's favorite Pokemon! When several Pikachu gather, their electricity could build and cause lightning storms."
    },
    "moves": [
      {
        "name": "Growl",
        "type": "normal"
      },
      {
        "name": "Quick Attack",
        "dp": 40,
        "type": "normal"
      },
      {
        "name": "Thunderbolt",
        "dp": 90,
        "type": "electric"
      }
    ],
    "buffs": [],
    "debuffs": []
  },
  "p2" : {
    "name" : "Ditto",
    "hp" : 206,
    "current-hp" : 206,
    "images": {
      "photo": "images/ditto.jpg",
      "typeIcon": "icons/normal.jpg",
      "weaknessIcon": "icons/fighting.jpg"
    },
    "info": {
      "id": "132",
      "type": "normal",
      "weakness": "fighting",
      "description": "Duncan's favorite Pokemon (he has an awesome painting of Ditto on his wall). It can transform into anything. When it sleeps, it changes into a stone to avoid being attacked."
    },
    "moves": [
      {
        "name": "Transform",
        "dp": 40,
        "type": "normal"
      }
    ],
    "buffs": [],
    "debuffs": []
  }
}
\end{lstlisting}
\hrule
\vspace{1em}

  You may assume that the \texttt{guid} and \texttt{pid} attributes returned are unique to the started game.

\subquestion*{Query 4: Play a Move}
  \textbf{Request Format:} \url{game.php} endpoint with \texttt{POST} parameters of \texttt{guid}, \texttt{pid}, and
  \texttt{movename}
  \newline
  \textbf{Request Type:} \texttt{POST}
  \newline
  \textbf{Returned Data Format:} \texttt{JSON}
  \newline
  \textbf{Description:} This query submits a move played by your
  Pokemon on the current turn and requires three parameters: \texttt{move} as your Pokemon's move name, \texttt{guid} as
  your unique game ID, and \texttt{pid} as your unique player id. The move name should be passed as
  an all-lowercase string, and if there are any spaces in the move name (e.g., "Quick Attack"), they
  should be removed when passed as a parameter (e.g., "Quick Attack" would be passed as
  "quickattack"). The game state is updated by applying that move's effects to
either player (depending on the specific effects of the move). The request will also call the
  opponent's move, which may update the health or buffs of your Pokemon. Upon success, the request
returns the current game state, including each player's current Pokemon status and the results of the
  two moves (yours and the opponents), as a JSON object.

An example return is given on the following page, where the guid provided is fictitious and you will need to provide the one retrieved from the previous \texttt{startgame} request:
\newpage
\textbf{Example Request:}
\newline
\texttt{POST} parameters of \texttt{guid=game\_12345abc}, \texttt{pid=player\_cfe67890}, and
\texttt{movename=quickattack}
\newline
\textbf{Example Output:}
\vspace{1em}
\hrule
\begin{lstlisting}
{
  "guid" : "game_12345abc",
  "results" : {
    "p1-move" : "Thunderbolt",
    "p2-move" : "Transform",
    "p1-result" : "hit",
    "p2-result" : "miss"
  },
  "p1" : {
    "name" : "Pikachu",
    "type" : "electric",
    "weakness" : "ground",
    "hp" : 80,
    "current-hp" : 60,
    "moves" : [
      {
        "name": "Brick Break",
        "dp" : 75,
        "type" : "fighting"
      },
      {
        "name": "Growl",
        "type" : "normal"
      },
      {
        "name": "Quick Attack",
        "dp" : 40,
        "type" : "normal"
      },
      { 
        "name": "Thunder",
        "dp" : 80,
        "type" : "electric"
      }
    ],
    "buffs" : [],
		"debuffs" : []
  },
  "p2" : {
    "name" : "Ditto",
    "type" : "normal",
    "weakness" : "fighting",
    "hp" : 60,
    "current-hp" : 20,
    "moves" : [
      {
        "name": "Transform",
        "dp" : 40,
        "type" : "normal"
      }
    ],
    "buffs" : [],
		"debuffs" : ["attack", "attack"]
  }
}
\end{lstlisting}
\hrule
\end{question}
\end{document}
